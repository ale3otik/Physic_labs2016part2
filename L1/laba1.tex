\documentclass[11pt]{article}
% Эта строка — комментарий, она не будет показана в выходном файле
\usepackage{ucs}
\usepackage[utf8x]{inputenc} % Включаем поддержку UTF8
\usepackage[russian]{babel}  % Включаем пакет для поддержки русского языка
\usepackage{amsmath}
\usepackage{mathtools}
\usepackage{amssymb}
% \usepackage[dvips]{graphicx}
% \graphicspath{{noiseimages/}}
\usepackage[pdftex]{graphicx}


% Параметры страницы: 1см от правого края и 2см от остальных.


\hoffset=0mm
\voffset=0mm
\textwidth=180mm        % ширина текста
\oddsidemargin=-6.5mm   % левое поле 25.4 - 5.4 = 20 мм
\textheight=240mm       % высота текста 297 (A4) - 40
\topmargin=-15.4mm      % верхнее поле (10мм)
\headheight=5mm      % место для колонтитула
\headsep=5mm          % отступ после колонтитула
\footskip=8mm         % отступ до нижнего колонтитула


\begin{document}
    \author {Зотов Алексей 497 гр.}
    \title {Лабораторная работа 2.1 \\  Определение $C_{p}/C_{v}$ по скорости звука в газе}
    \maketitle{}   

    \indent
    \textbf{Цель работы:}
         \begin{enumerate}
         \item Измерение частоты колебаний и длины волны при резонансе звуковых колебаний в газе, заполняющем трубу. 
         \item Определение показателя адиабаты по скорости звука с помощью уравнения состояния идеального газа.
         \end{enumerate}
    \indent
        
        \textbf{В работе используются:} звуковой генератор; электронный осциллограф; теплоизолированная труба; баллон со сжатым углекислым газом; газгольдер.

        

\end{document}