\documentclass[10pt]{article}
% Эта строка — комментарий, она не будет показана в выходном файле
\usepackage{ucs}
\usepackage[utf8x]{inputenc} % Включаем поддержку UTF8
\usepackage[russian]{babel}  % Включаем пакет для поддержки русского языка
\usepackage{amsmath}
\usepackage{amssymb}
\usepackage{mathtools}

\hoffset=0mm
\voffset=0mm
\textwidth=170mm        % ширина текста
\oddsidemargin=-0mm   % левое поле 25.4 - 5.4 = 20 мм
\textheight=240mm       % высота текста 297 (A4) - 40
\topmargin=-15.4mm      % верхнее поле (10мм)
\headheight=5mm      % место для колонтитула
\headsep=5mm          % отступ после колонтитула
\footskip=8mm         % отступ до нижнего колонтитула



% \textwidth=180mm    
% \oddsidemargin=-10mm 

\title{Лабораторная работа № 2.2: {\it Изучение затухающих колебаний в колебательном контуре.}}
\author{Зотов Алексей, 497}
\date{\today}

\begin{document}

\maketitle
\textbf{Цель работы:} Изучение параметров и характеристик колебательного контура.

\textbf{В работе испольуются:} \small{генератор звуковых сигналов, осциллограф, модуль с колебательным контуром ФПЭ–10, преобразователь импульсов ФПЭ–08, источник питания, магазин сопротивлений.}

\textbf{Ход работы:}
    \begin{enumerate}
    \item .
    \end{enumerate}
\end{document}